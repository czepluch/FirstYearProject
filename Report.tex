\documentclass[a4paper,11pt]{article}
\usepackage[T1]{fontenc}
\usepackage{inputenc}
\usepackage{amsfonts}
\usepackage{graphicx}
\usepackage{bm}
\usepackage{varioref}
\usepackage[english]{babel}
\usepackage{hyperref}
\newcommand{\field} [1] {\mathbb{#1}}
\begin{document}

\begin{titlepage}
\centering \parindent=0pt
\newcommand{\HRule}{\rule{\textwidth}{1mm}}
\vspace*{\stretch{1}} \HRule\\[1cm]\Huge\bfseries
Krak-vejkort\\[0.7cm]
\large Visualiseringen\\[1cm]
\HRule\\[4cm]  \large af \\Jacob Stenum Czepluch (jstc@itu.dk), \\Niels Liljedahl Christensen (nlch@itu.dk), \\Mikkel Larsen (milar@itu.dk), \\Sigurt Bladt Dinesen (sidi@itu.dk) \\
\vspace*{\stretch{2}} \normalsize %
\begin{flushleft}
IT-University\\
Copenhagen\\
First year project\\
Rasmus Pagh\\
\today \end{flushleft}
\end{titlepage}

\tableofcontents
\pagebreak

\pagebreak
\section{Introduction}

This will be our introduction to this small report\ldots


This is a section. Use it. Love it.
This is an example of a long text
This is an example of a long text
This is an example of a long text
This is an example of a long text
This is an example of a long text
This is an example of a long text
This is an example of a long text

Then we have a new line with indent
This is an example of a long text
This is an example of a long text
This is an example of a long text
This is an example of a long text

\pagebreak
\section{Design choices} % (fold)
\label{sec:Design choices}
Here we should write something about our design choices.

\subsection{Outline}
\begin{description}
	\item[MVC] \hfill \\
	Great!
	\item[Data structure] \hfill \\
	QuadTree
	\item[Visualisation] \hfill \\
	Platform
	
	How everything is drawn
	
	How the user interacts
	
	Types of the Krax-data
\end{description}
% section Design choices (end)

\pagebreak
\section{Implementation} % (fold)
\label{sec:Implementation} % This sections should describe our implementation
The implementation of the application consists of four different packages; the Model, View and Controller packages used as in the MVC design pattern, and a Global package storing global fields to be accessed and modified from all other packages.


\subsection{Controller package} % This is a simple description of the implementation of the Controller class.
The Controller package consists solely of the Controller class, which is both the main class (it has the main method run when the application starts), and it is the link between the Model and View packages handling the flow of data between the two. When a change is made by the user, the View calls a method in the Controller once again updating the graphical user interface according the both the input from the user and the data stored in the Model.

\subsection{Global package} % Contains all the global values used from all around the application
This package contains only the MinAndMaxValues class which has fields that needs to be accessed from the entire application. These fields include initial values such as the current "viewbox", min and max values for x- and y-coordinates, limits for when the different types of road segments are drawn etc. It also contains methods for checking whether are not the current viewbox results in a need for re-filtering the data to be drawn. The class is statically imported by all classes needing to access this information.

\subsection{Model package} % The description of the Model package is more complicated and consists of descriptions of several other classes.
The Model package consists of all the classes managing data storage, filtering, and conversion.

\subsubsection{Model class} % Makes use of the rest of the classes in the Model package. The front-end class.
The Model class is the front-end class of the Model package (the only class which is directly connected to the Controller class). This is where the data structure is stored in a field and where the methods for filtering and converting data are called. The data structure is stored with the type DataStructure, which is an interface allowing us to easily switch between data structures, as long as they implement this interface.

\subsubsection{XMLReader class} % Reads in the data from an XML file of krax format and adds the content to a given data structure
This class reads in data from an XML file of the KRAX format and converts it to instances of the Edge class (a simple class representing and edge on the roadmap), which then are added to a given data structure.

The XMLReader makes use of an external library, xom (
\url{www.xom.nu}
), for reading the XML data.

\subsubsection{QuadTreeDS class} % The data structure of the application. Consists of several other classes to be explained here
The QuadTreeDS is the basis of the entire application. It is in an instance of this all data is stored after being read by the XMLReader class. In order for it to be used as our data structure, it implements the DataStructure interface.

The class consists of four instances of the QuadTree class (one for each type of road segment). A QuadTree consists of nodes, which has an x- and a y-coordinate (stored as doubles) and a reference to an Edge object. Each QuadTree contains all edges of a given type. Each Edge object is stored twice; both referenced to by the start- and end-coordinates of the edge.

Inserting a node into a QuadTree is done recursively; the given node is compared to root node, deciding to which of the four children of the root the given node is to be compared to next. This continues until a null-reference / a leaf is found.

Retrieving information is done using an instance of the Interval2D class (representing a rectangle), which again consists of to instances of the Interval class (representing a line). This too is done recursively; it is checked whether the coordinates of the root node is within the given rectangle. If it is, it is added to a given collection of edges. It is then checked which of the subtrees might contain nodes within the rectangle, and for each that match, the same method is invoked, now with each of the matching children as the root. The call returns at null references.

Out implementation of the Quadtree (including the Interval and Interval2D classes) are heavily based upon implementations from
\url{algs4.cs.princeton.edu}. 

\subsubsection{FormatConverter class} % Converts from the data type pulled out of the data structure to the data type needed by the View package
Converts from the data type pulled out of the data structure to the data type needed by the View package.

\subsubsection{KraxToXMLConverter class} % Not directly part of the implementation, but the class used for converting the data from Krak to the krax XML format
Not directly part of the implementation, but the class used for converting the data from Krak to the krax XML format.

\subsection{View package} % Consists of all the classes handling the graphical user interface
Consists of all the classes handling the graphical user interface.

\subsubsection{View class} % Front end class in the MVC pattern. Contains the rest of the (non-static) GUI classes. Implements the MapListener interface. The overall class structure of the View package
Front end class in the MVC pattern. Contains the rest of the (non-static) GUI classes. Implements the MapListener interface. The overall class structure of the View package.

\subsubsection{MapPanel class} % Draws lines according to input data
Draws lines according to input data.

\subsubsection{ZoomHandler class} % Handles all the zooming
Handles all the zooming.

\subsubsection{DragHandler class} % Handles all the dragging
Handles all the dragging.

\subsection{Outline}
\begin{description}
	\item[Model] \hfill \\
	The Edge data type
	
	XMLReader
	
	QuadTree
	
	FormatConverter
	
	/KraxToXMLConverter
	\item[View] \hfill \\
	The overall class structure
	
	The use of the Observer design pattern
	
	Mouse and window listeners
	\item[Controller] \hfill \\
	The method calls from and to the map / data structure
\end{description}
% section Implementation (end)

\pagebreak
\section{Discussion} % (fold)
\label{sec:Discussion}
In this sections we will discuss what could have been done better, and/or what we think we have done right.

\subsection{Outline}
\begin{description}
	\item[Limitations] \hfill \\
	
	\item[Possible improvements] \hfill \\
	
	\item[Data structure discussion] \hfill \\
	QuadTree vs KD-tree
	
	Balanced vs unbalanced QuadTree
\end{description}
% section Discussion (end)

\pagebreak
\section{Conclusion} % (fold)
\label{sec:Conclusion}
This section will make a quick summary and conclusion on out project so far.
% section Conclusion (end)

\subsection{Test 2}

This is a subsection

\end{document}
