Apart from the typical improvements to the trie structure, there are a few issues and possible improvements that are specific to this project. These will be discussed here.

Firstly, a prefix search returns \underline{all} Strings in the trie with a matching prefix.
This is standard behaviour for a trie, but the \texttt{View} class discards all but five first matches. If the trie had done this itself, it would have improved the performance of prefix searches, as a search could end its descending through the tree after five hits, instead of exhausting the sub trees of matching nodes. As mentioned in the Discussion (\ref{sec:Trie.class} page \pageref{sec:Trie.class}),
we have largely disregarded optimizing the trie, because it is effective enough for our purpose.

The second issue is perhaps a bit more grave; the city locations are quite arbitrary.
The id for city entries in the trie are guaranteed to point to a node that is within the zip code of the city, nothing more. It is trivial, though cumbersome, to improve the choices for these location, so we have down prioritized it to a point were we have not gotten around to it.
If the program were to be used in real life application, it would probably have been a significant issue, but for the purpose of this project we have deemed it to be of small importance.

The code for the search trie bears the mark of being an ad-hoc solution. In order for the class to honour the OOP principle of reusability, the search key and return value should be parameterized at runtime, and the constructors for building the trie from a file should have taken a field separator as parameter rather than using the hardcoded ';' character.
