We have been experiencing some disappearing roads while viewing the map at a zoom level where we see all road types. This is due to the way our data structure is made, and we did not discover this problem in time to change the data structure and fix the problem entirely. Fortunately, it is only very long edges that disappear at certain zoom levels. The only edges that we have really experienced this problem with, are the ferry routes. But since then, we have decided on always showing all ferry road segments, eliminating that part of the issue.

When a road disappears, it happens because none of the nodes of the edge are in the "viewbox" of the program. When we zoom in, we need to know which roads to draw in the frame. This is done by checking if the coordinates of an edge's nodes are within the "viewbox" or its buffer zone. If neither of the nodes are within the visible part of the map, the edge is not drawn. So naturally, when a very long edge should be drawn in the "viewbox", but does not have any of its nodes in the view, it is not drawn. 
