Our current data structure consists of four quad trees, one per type of edge stored. This means that the application has to search through several quad trees if several types are to be drawn. It has both advantages and disadvantages. When all types are to be displayed, the four quadtrees are not as optimal as having only one quadtree, since all the trees have to be searched through, and having everything in the same tree would reduce the overall height of the tree(s). But on the other hand, having several quadtrees is better for situations where not all types are to be displayed, since only the quadtrees containing the needed types are searched through, not spending time on finding, but not using, edges of the wrong type.

Since all types are not displayed most of the time, we have decided on using four quad trees instead of one, lowering the performance in the situations where the user views all road types, but increasing performance in other situations and thereby the overall performance.