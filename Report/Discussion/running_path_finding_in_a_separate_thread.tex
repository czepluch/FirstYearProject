For quite a long period we had great trouble making the path-finding searches using Dijksta's algorithm. A single search could easily take more than 30 seconds to finish. We had very a lot of trouble figuring out what caused this slow search. We tried implementing the A* algorithm instead, but without any luck. Due to this slow search we started looking at the opportunity to make the path-finding run in a separate thread, giving the user the possibility to still navigate and explore the map. 

This was, however, after a meeting with Rasmus Pagh, not a problem anymore, since we found out that the priority queue that our Dijkstra implementation used was very slow and actually running in linear time when removing things. By deleting the part of the implementation where the non-fastest roads are removed from the priority queue, our path-finding search was now done in less than 0.5 seconds, and we did no longer see any reason to run the path-finding in a separate thread. Deciding not to remove the these non-fastest roads anymore results in a slight increase in memory usage, but we believe that the speed we have gained by doing so more than makes up for it. Also, since these roads are not the fastest between two given points, they will never be looked at again and are, as far as the algorithm is concerned, essentially gone.