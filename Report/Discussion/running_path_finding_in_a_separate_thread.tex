For quite a long period, we had great trouble making the path-finding searches using Dijksta's algorithm. A single search could easily take more than 30 seconds to finish. We had a lot of trouble figuring out what caused this slow search. We tried implementing the A* algorithm instead, but without any luck. Due to this slow search, we started looking at the opportunity to make the path-finding run in a separate thread, giving the user the possibility to still navigate and explore the map while searching for a path. 

This was, however, after a meeting with Rasmus Pagh, not a problem any more, since we found out that the priority queue that our Dijkstra implementation used was very slow and actually running in linear time when removing things. By changing the implementation to use another implementation of a priority queue, our path-finding search was now done in less than half a second in many cases, and we did no longer see any reason to run the path-finding in a separate thread.

It turned out that our previous priority queue had some problems finding the correct path sometimes. Due to this we decided to implement another priority queue than Java's standard priority queue. The new priority queue is taken from the algs4.cs.princeton.edu homepage by Robert Sedgewick and Patrick Wayne. Using this priority queue fixes our problem of not always finding the correct path.
