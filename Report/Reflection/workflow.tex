As an unwritten constitution, we have begun most meetings by writing an agenda (these can be found in appendix \ref{sec:Diary} page \pageref{sec:Diary}).

In adherence to our constitution (as seen in appendix \ref{sec:Constitution} on page \pageref{sec:Constitution})
we have alternated between two work forms: 
\begin{itemize}
	\item Plenum - working together or separately, but doing so while sitting in the same room or s in a Skype session.
	\item Distributed - between meetings, tasks are appointed to individuals or subgroups and finished before the next meeting.
\end{itemize}
Working in plenum enabled us to work as a group, sparring with each other on problems and decisions, while distributing work gave each member a little more freedom to decide when to work on the project.
It did, however, come with a price. A couple of times (most notably when interfacing path searches with the GUI) we did not think to take the time to agree on an API before going home. The result was that when we would meet again, some fragments of code would be completely incompatible and we had to spend time on refactoring it.

Appart from the different work forms, we have not paid much attention to the written constitution. The constition primarily adresses how to solve problems with disagreement, and we have not come to the point where it was necessary to use it.
