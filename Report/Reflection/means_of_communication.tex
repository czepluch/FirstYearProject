We have used a Facebook group to communicate between scheduled meetings. While Facebook provides a very passive form of communication, the use of which can be risky if a message is urgent or important, the problems it has caused 
have been insignificant. The reason for this seems to be our habit of ending each meeting by scheduling the next meeting and assigning work to each member of the group. This way, not a lot of extra communication has been necessary, diminishing the need for posting on the Facebook group-wall. In addition to that, the Facebook group has been supplemented with us having each others phone numbers, meaning that if a discussion was taking place on Facebook, a text message could be send to the group members not participating, letting them know of the situation.

Quite a few of our meetings have been held via Skype, rather than having real-life meetings. The primary reason for this is that one member of the group lives far away from ITU, our usual rendezvous point, and it simply did not make sense for him to travel that far for short meetings.
Contrary to the Facebook group, Skype has proven to be a poor choice for us. We had problems with volatile internet connections and generally found that, for internet meetings to work for us, we needed better tools for cooperatively editing and viewing documents and code. In short; Skype was not the right tool for the job, and poor internet connections did not help to ease the process.

For future reference, it seems recommendable to predefine the communicative means the group wishes to make use of as early as possible. As for internet meetings; other tools exists, but it might be easier to just meet in person.