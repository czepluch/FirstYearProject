\section{Introduction}

This report covers the our first-year project at the MSc in Software Development at the IT-University of Copenhagen. This project has been made in March, April, and May in the year 2012. 

To make this project, we were randomly divided into groups of size four to six. Our group, group 8, consists of four students. 

The main purpose of this project is to visualise road-map data provided by Krak, as well as to show locations and provide directions between two given points on the map. The data is to be visualised as a map displaying all roads within a given square area of Denmark.
It must be possible to zoom in and out on a desired part of the map. In addition, the level of detail should follow the zoom level. It is also required that the window in which the map is displayed can be resized. In the description of our assignment, a few optional, possible additions to the program were suggested. We have also added some features of our own to the program.

There are worksheets corresponding to the work we have made at, and due to every meeting. The worksheets also contain information about agreements formed in the introductory phase of the project, as well as agreements formed throughout the project.

Our advisor throughout this project has been Rasmus Pagh, from whom we have received both guidance and relevant lectures. We have also received feedback from our tutor, Filip Sieczkowski, who has been a great help to us.

Furthermore we would like to thank Robert Sedgewick and Kevin Wayne from Princeton University for providing well written and easy-to-understand code and algorithms at \url{algs4.cs.princeton.edu} that we have chosen to use and rewrite for our data structure in this project.