A few of the classes of the application contains code which is not entirely written by ourselves. The classes \texttt{QuadTree}, \texttt{Interval}, \texttt{Interval2D}, and \texttt{IndexMinPQ} are classes from \url{http://algs4.cs.princeton.edu/}, which are then modified to fit the needs of our application. The exact URLs / links to each of the classes are to be found as a comment within each of the classes, just below the imports. Our implementation of the \texttt{TernaryTrie} class is also inspired by an implementation from the  aforementioned website.

The \texttt{QuadTree}, \texttt{Interval}, and \texttt{Interval2D} classes were originally generic of a type extending \texttt{Comparable}. Our modifications extend to changing the generic type to be of the simple type \texttt{double}, allowing us to do comparisons using the \texttt{>}, \texttt{<}, and \texttt{==} operators and avoid auto-boxing and -unboxing.

Actually, the classes were not compliant as they were first downloaded (or more exactly, the \texttt{Interval2D} class was not compliant with the others). But few modifications removed the issues. The logic of the code is still more or less exactly the same as it originally was.

We found inspiration to our implementation of Dijkstra's algorithm (in the \texttt{Dijkstra} class) at \\ \href{http://en.literateprograms.org/Special:DownloadCode/Dijkstra\%27s_algorithm_(Java)}{en.literateprograms.org} \\ (\url{http://en.literateprograms.org/Special:DownloadCode/Dijkstra\%27s_algorithm_(Java)}).