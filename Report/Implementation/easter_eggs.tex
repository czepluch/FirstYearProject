The application includes two hidden features, so called easter eggs.

The first is concerning the input typed in by the user. The application allows the input being written as ASCII characters in binary, whereas the application will translate the input numbers to the corresponding address (similar to a regular search), once a search is executed.

The binary conversion, however, has its limitation. Since only ASCII characters are allowed, characters such as the "Nordic" letters are not allowed, or more precisely they are not possible to write. Also, the automatic text completion does not update the list of possible addresses while a binary input is typed by the user. This is, because the binary addresses are not stored withing the \texttt{TernaryTrie} class, but it is translated once the \texttt{Find}-button is pushed.

The second easter egg is that if the user types in the string "\textit{Rainbow Road}", trips will from there on be drawn using six different colors. For each road segment on the trip, a new color is used. Typing "\textit{Ordinary Road}" will reset the application to display trips in the default manner once again.