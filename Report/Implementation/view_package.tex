The \texttt{View} package contains all the classes managing the graphical user interface.


\subsubsection{View class} % Front end class in the MVC pattern. Contains the rest of the (non-static) GUI classes. Implements the MapListener interface. The overall class structure of the View package
The \texttt{View} class is the front-end class of the \texttt{View} package in the MVC design pattern. It contains an instance of the \texttt{MainFrame} class, which is the basic \texttt{java.swing} GUI (the window to be displayed), which then again contains an instance of our custom panels, the \texttt{MapPanel} and the \texttt{SearchPanel}.

The \texttt{View} itself implements the interfaces \texttt{MapListener} and \texttt{SearchListener}. An instance of the \texttt{MapListener} is stored in both the \texttt{MainFrame} and the \texttt{MapPanel} classes. This allows for these classes to invoke a method in the \texttt{View}, telling it that changes has been made, which then invokes a similar method in the \texttt{Controller}, which then updates the GUI according the the changes. An instance of the \texttt{SearchListener} is stored in the \texttt{SearchPanel}, giving the same possibilities.

\subsubsection{MinAndMaxValues class}
This class is an exception from our use of the MVC pattern. It contains public, static fields that need to be accessed from the entire application. These fields include initial values such as the current "viewbox", minimum and maximum values for x- and y-coordinates, definitions for when the different types of road segments are drawn etc. It also contains methods for checking whether or not the current viewbox results in a need for re-filtering the data to be drawn. The class is statically imported by all classes needing to access this information.

\subsubsection{MapPanel class} % Draws lines according to input data
The \texttt{MapPanel} class extends the \texttt{java.swing.JPanel} class and functions as a panel with extended functionality and with an overridden \texttt{paint} method.

The \texttt{MapPanel} stores an \texttt{int[][][]} (as generated by the \texttt{FormatConverter}), from which it draws lines of the canvas, each corresponding to an edge stored in the \texttt{Model}. It also stores an instance of the \texttt{MapLocation} and \texttt{Trip} class (which both can be null), which are then drawn similarly to the roads, given that they are not null. The \texttt{MapLocation} is displayed as a filled circle, whereas the trip is displayed by drawing each of the road segments of which it consists.

The \texttt{MapPanel} also has two listeners from the \texttt{java.awt} library; a \\\texttt{MouseWheelListener}, which invokes a static method of the \texttt{ZoomHandler} class when the user scrolls on the panel, sending data about the mouse coordinates and the scrolled amount, and a \texttt{MouseMotionListener}, which invokes a static method of the \texttt{DragHandler} class when the user drags the mouse on the panel, letting the \texttt{DragHandler} know how far the mouse has been dragged (and along which axes).

\subsubsection{ZoomHandler class} % Handles all the zooming
The \texttt{ZoomHandler} handles all zooming. When a call is received (from the \texttt{MapPanel}), signaling that the user wishes to zoom out, there are two possible outcomes: If the current viewbox is not too close to the maximum width and height values; it zooms out, maintaining the current center of the viewbox. Otherwise, if the viewbox is near its extrema, zooming is done with the viewbox bound to the borders imposed by the maximum values.

Subsequently to zooming in, a call is made to the \texttt{DragHandler} class, moving the viewbox towards the current location of the cursor.

Zooming is simply a matter of changing the global values indicating what is being displayed (in the \texttt{MinAndMaxValues} class) / changing the size of the viewbox and then invoking a repaint of the \texttt{MapPanel}.

\subsubsection{DragHandler class} % Handles all the dragging
The \texttt{DragHandler} class, like the \texttt{ZoomHandler} class, changes the viewbox according to input data (drag amount and direction) and according the the maximum values that definine the extrema of the viewbox.

\subsubsection{SearchPanel class}
Like the \texttt{MapPanel} class, the \texttt{SearchPanel} class extends \texttt{javax.swing.JPanel}.

The \texttt{SearchPanel} contains components used for searching for addresses; two \texttt{JTextFields} allowing the user to input up to two addresses, two \texttt{JLists} for displaying proposed addresses to the user, two \texttt{JButtons}, one for executing a search for either a location corresponding to an address, or a search for a trip from one given address to another, two \texttt{JRadioButtons} in a \texttt{ButtonGroup} for toggling between searching for either the fastest or the shortest trip.

The \texttt{SearchPanel} else has several \texttt{JLabels}. Each component used for user interaction is labeled with explaining text. Below these components, there are label displaying the following info about the trip:
\begin{itemize}
 \item The distance of the trip in meters.
 \item The computed time of the trip in minutes.
\end{itemize}
given a such search has been executed recently (and a search for a location has not been made since).

All component listeners on the SearchPanel (be it ActionListeners, DocumentListeners etc.) either only affect things within the \texttt{SearchPanel} class or signals to the \texttt{SearchListener} (in this case the \texttt{View} class) to take action according to the input given by the user.

\subsubsection{AddressParser class}
The \texttt{ParseAddress} method is the basis of the \texttt{AddressParser} class. It uses regular expressions and pattern matching to parse an input address. The basic idea is that an input address can consist of up to three different "kinds" of parts:
\begin{itemize}
	\item Zero, one, or two city name / street name parts consisting of the characters allowed in them (i will reference to this as "\textit{name}").
	\item Zero or one zip code part consisting of digits (i will reference to this as "\textit{zip}").
	\item Zero or more parts containing the characters not allowed elsewhere, which allows the other parts to be determined from each other (i will reference to this as "\textit{other}").
\end{itemize}
The input \texttt{String} is matched with up to three different patterns (i will use the $\sharp$ as a separator. The words \underline{underlined} must be present, the rest are optional):
\begin{itemize}
	\item Pattern with the zip code last: \\
		\textit{other$\sharp$\underline{name}$\sharp$other$\sharp$name$\sharp$other$\sharp$zip$\sharp$other}
	\item Pattern with the zip code first \\
	\textit{other$\sharp$\underline{zip}$\sharp$other$\sharp$name$\sharp$other$\sharp$name$\sharp$other}
	\item Pattern with the zip code in the middle \\
	\textit{other$\sharp$\underline{name$\sharp$other$\sharp$zip}$\sharp$other$\sharp$name$\sharp$other}
\end{itemize}
If the first pattern matches, the city name, street name and zip code (or at least the parts found) are parsed to the format \textit{city name$\sharp$zip code$\sharp$street name}. If it doesn't match, the parser tries matching the input with the next pattern, parsing to the same format, and so on. If none of the patterns match, \texttt{null} is returned.

The \texttt{ParseAddressLive} method makes use of the \texttt{ParseAddress} method, but it cleans up the result for it to be optimal for doing a prefix search in the \texttt{TernaryTrie}. It removes any "empty" parts of the address which are in either the beginning or the end of the parse address. An example is the parsed address \textit{$\sharp$1234$\sharp$}, which will be cleaned to being simply \textit{1234}.

\subsubsection{TernaryTrie class}
The search trie uses an internal node representation of characters to build a ternary tree, where each branch represents a street or city.
The tree structure makes it possible to search for a given string in time that is $\sim\ln(N)$ for a search miss and  $\sim K+1$ for a search hit, where K is the length of the string to search for.
In practise, on a 5 year old computer with a 2.4 Ghz intel dual-core processor, we have measured the prefix searches to take about one hundredth of a second on our trie data set.

The trie  provides a simple API for searches:
\begin{itemize}
	\item String get(String) returns the node id associated with the location denoted by ta given String, or null if the search is a miss
	\item Iterable<String> startsWith(String) returns all Strings that are prefixed with the given String, or null if no such Strings are found.
\end{itemize}

The code for the search trie bears the mark of being an ad-hoc solution. In order for the class to honour the OOP principle of reusability, the search key and return value should be parameterized at runtime, and the constructors for building the trie from a file should have taken a field separator as parameter rather than using the hardcoded ';' character.