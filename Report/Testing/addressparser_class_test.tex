In the \texttt{AddressParser} class, the \texttt{parseAddress} method has been tested. The method has been tested using two different tests; a whitebox test, testing all the possible conditional outcomes of the method, and a blackbox test, testing different addresses of different types, testing the stability of the method.

The two different tests share some of the same input data sets. The input data sets of both tests can be seen in the shared expectancy table in the appendix, section \ref{sec:AddressParser test appendix} on page \pageref{sec:AddressParser test appendix}.

The whitebox test consists of only four different inputs / input properties, because the methods only has four possible "outcomes" / "method flows"; either one of the three regular expressions for matching the address matches (represented by data set D1, E1, and F1), or the given input could not be interpreted as an address by the parser (no match), which is represented by data set G1.

The whitebox test is not a test for the stability of the method (the blackbox test is), but a test of whether all of the conditional parts of the method are reachable. Since all four outputs of the corresponding data sets are equal to what the results are supposed to be, the test validates the reachability of all parts.

The blackbox test is a test of the stability of the method. The test contains input properties representing various possible ways to enter an address, valid or not. Due to the nature of the input, the amount of possible inputs are more than enormous. When doing the test, our focus has been including as many possible ways to enter an address,   while having the description of valid input addresses from the Design choices section on page \pageref{sec: What is a valid address}. This includes both addresses which by the definition are valid and invalid.

A limit to the test is that is is made by a single person. It is very likely that having more people included in the testing would show missing input properties, which therefore currently has not been tested. We do, however, feel that the test includes a wide variety of inputs, and some addresses might be found, which cause the method to not produce the expected result, the method is proven to be stable in many cases.

In addition to the blackbox test, the \texttt{parseAddress} method has been tested severely by us while testing the general user interface of the application, since the method is invoked each time a character is written in one of the input fields.