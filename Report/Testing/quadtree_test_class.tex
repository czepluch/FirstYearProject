The purpose of this test is to test whether \texttt{QuadTree} is correctly built. This is done by inserting a number of edges and nodes into a \texttt{QuadTree} and checking whether the \texttt{query2D} method (which performs a range search in a given two-dimensional interval) finds the correct edges. The input data set is small enough to give a good overview of what each query should produce, but still large enough to allow a certain diversity in the test cases. If the data set was very small, several tests might yield the same results, making it difficult to tell if this is by accident, or if the query actually produced the right outcome. The expectancy table of the test can be found in the appendix, section \ref{sec:Quadtree class} on page \pageref{sec:Quadtree class}.

As with the \texttt{Dijkstra} class, we can not be certain that no special cases exists in which the class does not perform as it is supposed to. But the test along with general usage of the application give us an indication of its stability.