The \texttt{GVertex} and \texttt{Vertex} classes both have a \texttt{distance} method computing and return the distance between two given vertices. The only different is the type of vertices to be given as parameters (of type \texttt{GVertex} or \texttt{Vertex}). Therefore, the methods also share a the expectancy table of a single blackbox test, which can be found in the appendix, section \ref{sec:GVertex and Vertex classes} on page \pageref{sec:GVertex and Vertex classes}.

The nature of the input as a little more complicated than with the methods of the \texttt{Coordinates} class, but we believe then input properties represented by the expectancy table are representative for, if not all, most types of input. In order for take in account possible rounding errors, all outputs are rounded to the nearest $1/10000$ before compared to the expected output (which is also rounded). When the methods are working with UTM coordinates, the results are the distance being rounded to nearest $0.1 mm$, which we believe is more than exact enough for the purpose of displaying the roads as done by the application. The expected results are computed on a calculator in which we trust and rounded by hand. All tests are passed.

\#!!! Code duplication? Where is GVertex used?

