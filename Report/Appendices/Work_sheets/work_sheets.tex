\subsection*{Shortest Path}
\begin{tabular}{ | p{12cm} | }
	\hline
	\textbf{Title:} \\ 
	Compute the shortest / fastest path \\ 
	\\ \hline
	\textbf{Description:} \\
	Make a graph representation, allowing the program to compute the shortest / fastest path from one given point to another \\
	\\ \hline
	\textbf{Design decisions:}
	\begin{itemize}
		\item Use a non-simplified graph (at least at first), allowing us to easily draw the results
		\item Use Dijkstra’s algorithm
		\item The output is an object of type Trip:
		\begin{itemize}
			\item Contains info about the time and distance of the entire trip / path
			\item Contains a list a custom edge type, containing info about:
			\begin{itemize}
				\item From and to coordinates (in UTM 32)
				\item Distance of the edge
				\item Time of the edge
				\item Enum of type Direction
			\end{itemize}
		\end{itemize}
		\item Allows applying a filter of what types of road segments are to be included
	\end{itemize}
	\\ \hline
	\textbf{Implementation:} \\
	\\ \hline
	\textbf{Comments:} \\
	\\ \hline
\end{tabular}

\pagebreak
\begin{tabular}{ | p{12cm} | }
	\hline
	\textbf{Title:} \\
	Expand the FormatConverter \\
	\\ \hline
	\textbf{Description:} \\
	Expand the FormatConverter to convert the coordinates in the the Trip type to pixels \\
	\\ \hline
	\textbf{Design decisions:} \\
	\\ \hline
	\textbf{Implementation:} \\
	\\ \hline
	\textbf{Comments:} \\
	\\ \hline
\end{tabular}

\pagebreak
\begin{tabular}{ | p{12cm} | }
	\hline
	\textbf{Title:} \\
	Draw the shortest / fastest path \\
	\\ \hline
	\textbf{Description:} \\
	Draw a given list of edges and a path \\
	Show info about the trip \\
	\\ \hline
	\textbf{Design decisions:}
	\begin{itemize}
		\item Draw the path / edges
		\item Show a list containing info about both the entire road and the segments
		\item The input is of type Trip
	\end{itemize}
	\\ \hline
	\textbf{Implementation:} \\
	\\ \hline
	\textbf{Comments:} \\
	\\ \hline
\end{tabular}

\pagebreak

\subsection*{Search for directions}
\begin{tabular}{ | p{12cm} | }
	\hline
	\textbf{Title:} \\
	The search GUI \\
	\\ \hline
	\textbf{Description:} \\
	Make it possible to search, both for a single "point" or a trip \\
	\\ \hline
	\textbf{Design decisions:}
	\begin{itemize}
		\item Two input text fields and a button
		\item If second button does not contain info, search for "point", else search for trip
		\item The panel containing user interaction components is to the left
	\end{itemize}
	\\ \hline
	\textbf{Implementation:} \\
	\\ \hline
	\textbf{Comments:} \\
	\\ \hline
\end{tabular}

\pagebreak
\begin{tabular}{ | p{12cm} | }
	\hline
	\textbf{Title:} \\
	Auto completion \\
	\\ \hline
	\textbf{Description:} \\
	Auto completion (and potentially spell-correction) of addresses \\
	\\ \hline
	\textbf{Design decisions:}
	\begin{itemize}
		\item We need to define the UI-wise implementation. None of the classes in Java's standard library (that we know of) encompass the visualisation of this concept
	\end{itemize}
	\\ \hline
	\textbf{Implementation:} \\
	Still strongly tied to making the design decisions \\
	\\ \hline
	\textbf{Comments:} \\
	There is an external library that does this, but the question is if we wouldn't rather do it ourselves \\
	\\ \hline
\end{tabular}

\pagebreak

\subsection*{Draw the landscape}
\begin{tabular}{ | p{12cm} | }
	\hline
	\textbf{Title:} \\
	Read in the XML file and store it \\
	\\ \hline
	\textbf{Description:} \\
	Read in the XML file so it can be parsed to the view \\
	\\ \hline
	\textbf{Design decisions:}
	\begin{itemize}
		\item Make a separate quad tree for the landscape, containing a custom polygon type
	\end{itemize}
	\\ \hline
	\textbf{Implementation:} \\
	\\ \hline
	\textbf{Comments:} \\
	\\ \hline
\end{tabular}

\pagebreak
\begin{tabular}{ | p{12cm} | }
	\hline
	\textbf{Title:} \\
	Draw the landscape
	\\ \hline
	\textbf{Description:} \\
	Draw the landscape beneath the road segments
	\\ \hline
	\textbf{Design decisions:} \\
	\\ \hline
	\textbf{Implementation:} \\
	\\ \hline
	\textbf{Comments:} \\
	\\ \hline
\end{tabular}