The \texttt{TernaryTrie} class is tested by running it methods on a small data set (\texttt{data.trie.test}). The following are the data sets and expectancy tables for the different tests:

\subsubsection{The data}
\begin{tabular}{ p{8cm} | p{2cm} }
	\textbf{Input property} & \textbf{Input data set} \\	
	\hline
	Street$\#$City;id & A1 \\
	Street$\#$Zip$\#$City;id & A2 \\
	City$\#$Zip$\#$Street;id & B1 \\
	City$\#$Street;id &  B2 \\
	City$\#\#$;id & C \\
\end{tabular}
\\
\\
\\
\begin{tabular}{ p{8cm} | p{2cm} }
\textbf{Input data} & \textbf{Input data set} \\
	\hline
	Fuglevej$\#\#$Lille Skensved;1 & A1 \\
	Lindingvadvej$\#\#$Tj\ae reborg;2 \\
	Kirkevang$\#\#$Stenl\o se;3 \\
	Kindertofte Skolevej$\#\#$Slagelse;4 \\
	\\
	M\aa gevej$\#6950\#$Ringk\o bing;5 & A2 \\
	Kalkbr\ae nderil\o bskaj$\#2100\#$K\o benhavn \O ;6 \\
	Over Isen Vej$\#7430\#$Ikast;7 \\
	\\
	N\ae stved$\#4700\#$Ahornvej;8 & B1 \\
	Randers C$\#8900\#$Lindholtsvej;9 \\
	Vedb\ae k$\#2950\#$Ved Vejen;10 \\
	\\
	Otterup$\#\#$Franketoftegyden;11 & B2 \\
	Kongens Lyngby$\#\#$Om K\ae ret;12 \\
	Kib\ae k$\#\#$Troldhedevej;13 \\
	\\
	K\o ge$\#\#$;14 & C \\
\end{tabular}