The \texttt{Quadtree} class has been tested using a small sample input, on which edge filterings with known expected outputs are run. The actual output is then compared to the expected output. \\ \\

\begin{tikzpicture}
	\draw [->] (0,0)--(0,10);
	\draw [->] (0,0)-|(10,0);
	
	\draw (1,-0.25)-- node [below, text height=15pt] {1} (1,0.25);	
	\draw (2,-0.25)-- node [below, text height=15pt] {2} (2,0.25);	
	\draw (3,-0.25)-- node [below, text height=15pt] {3} (3,0.25);	
	\draw (4,-0.25)-- node [below, text height=15pt] {4} (4,0.25);	
	\draw (5,-0.25)-- node [below, text height=15pt] {5} (5,0.25);	
	\draw (6,-0.25)-- node [below, text height=15pt] {6} (6,0.25);	
	\draw (7,-0.25)-- node [below, text height=15pt] {7} (7,0.25);	
	\draw (8,-0.25)-- node [below, text height=15pt] {8} (8,0.25);	
	\draw (9,-0.25)-- node [below, text height=15pt] {9} (9,0.25);
	
	\draw (-0.25,1)-- node [left, text width=15pt] {1} (0.25, 1);
	\draw (-0.25,2)-- node [left, text width=15pt] {2} (0.25, 2);
	\draw (-0.25,3)-- node [left, text width=15pt] {3} (0.25, 3);
	\draw (-0.25,4)-- node [left, text width=15pt] {4} (0.25, 4);
	\draw (-0.25,5)-- node [left, text width=15pt] {5} (0.25, 5);
	\draw (-0.25,6)-- node [left, text width=15pt] {6} (0.25, 6);
	\draw (-0.25,7)-- node [left, text width=15pt] {7} (0.25, 7);
	\draw (-0.25,8)-- node [left, text width=15pt] {8} (0.25, 8);
	\draw (-0.25,9)-- node [left, text width=15pt] {9} (0.25, 9);
	
	\draw (0,0)-- node [auto, swap] {e1} (4,3);
	\draw (0,0)-- node [auto] {e2} (6,8);
	\draw (4,2)-- node [auto, swap] {e3} (6,8);
	\draw (4,2)-- node [auto] {e4} (4,3);
	
	%I1
	\node [draw=none, fill=none] (I1) at (-0.4, -0.4) {I1};
	
	%I2
	\draw [dashed] (3,5)--(3,7);
	\draw [dashed] (3,7)--(7,7);
	\draw [dashed] (7,7)--(7,5);
	\node [draw=none, fill=none] (I2) at (7.5,4.5) {I2};	
	
	%I3
	\draw [dashed] (3,0)--(3,5);
	\draw [dashed] (3,5)--(8,5);
	\draw [dashed] (8,5)--(8,0);	
	\draw [dashed] (8,0)--(3,0);
	\node [draw=none, fill=none] (I3) at (3.5,6.5) {I3};
	
	%I4
	\draw [dashed] (0,0)--(0,8);
	\draw [dashed] (0,8)--(8,8);
	\draw [dashed] (8,8)--(8,0);
	\draw [dashed] (8,0)--(0,0);
	\node [draw=none, fill=none] (I4) at (1,7.5) {I4};
\end{tikzpicture}
\\ \\ \\
\begin{tabular}{ p{1.5cm} | p{3cm} | p{4cm} | p{1.5cm} }
	Interval & (x1, x2), (y1, y2) & Expected output & Actual output \\
	\hline
	I1 & (0,0), (0,0) & $Size = 2:$ \newline $e1$, $e2$ & $\surd$ \\
	I2 & (3,8), (0,5) & $Size = 4:$ \newline $e1$, $e3, $e4, $e4$ & $\surd$ \\
	I3 & (3,7), (5,7) & $Size = 0$ & $\surd$ \\
	I4 & (0,8), (0,8) & $Size = 8:$ \newline $e1$, $e1$, $e2$, $e2$, $e3$, $e3$, $e4$, $e4$ & $\surd$
\end{tabular}